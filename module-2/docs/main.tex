\documentclass[12pt,onecolumn]{article}
\usepackage[utf8]{inputenc} % UTF8 input encoding
\usepackage[T2A]{fontenc}   % T2A font encoding for Cyrillic script
\usepackage[russian]{babel} % Russian language support
\usepackage{listings}
\usepackage{float}
\usepackage{mathtools}
\everymath{\displaystyle}
\usepackage{listings} 
\usepackage[usenames]{color}
\usepackage{geometry}
\usepackage{verbatim}
\newcommand{\nparagraph}[1]{\paragraph{#1}\mbox{}\\}
\geometry{
  a4paper,
  top=25mm, 
  right=15mm, 
  bottom=25mm, 
  left=15mm
}

\begin{document}
\setcounter{tocdepth}{4}
\begin{center}
    Федеральное государственное автономное образовательное учреждение высшего образования "Национальный Исследовательский Университет ИТМО"\\ 
    Мегафакультет Компьютерных Технологий и Управления\\
    Факультет Программной Инженерии и Компьютерной Техники \\
    \includegraphics[scale=0.3]{image/itmo.jpg} % нужно закинуть картинку логтипа в папку с отчетом
\end{center}
\vspace{1cm}


\begin{center}
    \textbf{Модуль №2}\\
    по дисциплине\\
    \textbf{'Системы искусственного интеллекта'}
\end{center}

\vspace{2cm}

\begin{flushright}
  Выполнил Студент  группы P33102\\
  \textbf{Лапин Алексей Александрович}\\
  Преподаватель: \\
  \textbf{Авдюшина Анна Евгеньевна}\\
\end{flushright}

\vspace{6cm}
\begin{center}
    г. Санкт-Петербург\\
    2023г.
\end{center}

\newpage
\tableofcontents
\newpage

\section{Введение:}
\subsection{Описание целей проекта и его значимости.}
\section{Анализ требований:}
\paragraph{Описание лабораторных работ:}
\subparagraph{Лабораторная 4. Линейная регрессия }
Задание
\begin{itemize}
  \item Выбор датасетов: Студенты с нечетным порядковым номером в группе должны использовать про обучение студентов
  \item Получите и визуализируйте статистику по датасету (включая количество, среднее значение, стандартное отклонение, минимум, максимум и различные квантили).
  \item Проведите предварительную обработку данных, включая обработку отсутствующих значений, кодирование категориальных признаков и нормировка.
  \item Разделите данные на обучающий и тестовый наборы данных.
  \item Реализуйте линейную регрессию с использованием метода наименьших квадратов без использования сторонних библиотек, кроме NumPy и Pandas (для использования коэффициентов использовать библиотеки тоже нельзя). Использовать минимизацию суммы квадратов разностей между фактическими и предсказанными значениями для нахождения оптимальных коэффициентов.
  \item Постройте три модели с различными наборами признаков.
  \item Для каждой модели проведите оценку производительности, используя метрику коэффициент детерминации, чтобы измерить, насколько хорошо модель соответствует данным.
  \item Сравните результаты трех моделей и сделайте выводы о том, какие признаки работают лучше всего для каждой модели.
  \item Бонусное задание - Ввести синтетический признак при построении модели
\end{itemize}
\subparagraph{Лабораторная 5. Метод k-ближайших соседей}
Задание
\begin{itemize}
  \item Выбор датасета: Нечетный номер в группе - Датасет про диабет
  \item Проведите предварительную обработку данных, включая обработку отсутствующих значений, кодирование категориальных признаков и масштабирование.
  \item Реализуйте метод k-ближайших соседей без использования сторонних библиотек, кроме NumPy и Pandas. 
  \item Постройте две модели k-NN с различными наборами признаков:
      \begin{itemize}
        \item Модель 1: Признаки случайно отбираются .
        \item Модель 2: Фиксированный набор признаков, который выбирается заранее.
      \end{itemize}
  \item Для каждой модели проведите оценку на тестовом наборе данных при разных значениях k. Выберите несколько различных значений k, например, k=3, k=5, k=10, и т. д. Постройте матрицу ошибок.
\end{itemize}
\subparagraph{Лабораторная 6. Деревья решений}
\begin{itemize}
  \item Для студентов с четным порядковым номером в группе – датасет с классификацией грибов, а нечетным – датасет с данными про оценки студентов инженерного и педагогического факультетов (для данного датасета нужно ввести метрику: студент успешный/неуспешный на основании грейда)
  \item Отобрать случайным образом sqrt(n) признаков
  \item Реализовать без использования сторонних библиотек построение дерева решений (numpy и pandas использовать можно, использовать списки для реализации  дерева - нельзя)
  \item Провести оценку реализованного алгоритма с использованием Accuracy, precision и recall
  \item Построить AUC-ROC и AUC-PR (в пунктах 4 и 5 использовать библиотеки нельзя)
\end{itemize}
\subparagraph{Лабораторная 7.  Логистическая регрессия}
\begin{itemize}
  \item Выбор датасета: Датасет о диабете: Diabetes Dataset
  \item Загрузите выбранный датасет и выполните предварительную обработку данных. 
  \item Разделите данные на обучающий и тестовый наборы в соотношении, которое вы считаете подходящим.
  \item Реализуйте логистическую регрессию "с нуля" без использования сторонних библиотек, кроме NumPy и Pandas. Ваша реализация логистической регрессии должна включать в себя:
  \begin{itemize}
    \item Функцию для вычисления гипотезы (sigmoid function).
    \item Функцию для вычисления функции потерь (log loss).
    \item Метод обучения, который включает в себя градиентный спуск.
    \item Возможность варьировать гиперпараметры, такие как коэффициент обучения (learning rate) и количество итераций.
  \end{itemize}
  \item Исследование гиперпараметров: Проведите исследование влияния гиперпараметров на производительность модели. Варьируйте следующие гиперпараметры:
  \begin{itemize}
    \item Коэффициент обучения (learning rate).
    \item Количество итераций обучения.
    \item Метод оптимизации (например, градиентный спуск или оптимизация Ньютона).
  \end{itemize}
  \item Оценка модели: Для каждой комбинации гиперпараметров оцените производительность модели на тестовом наборе данных, используя метрики, такие как accuracy, precision, recall и F1-Score.
  \item Сделайте выводы о том, какие значения гиперпараметров наилучшим образом работают для данного набора данных и задачи классификации. Обратите внимание на изменение производительности модели при варьировании гиперпараметров.
  \end{itemize}
Проект ориентирован на развитие ключевых компетенций в области информационных технологий и имеет практическое применение в помощи пользователям при выборе видеоигр, что делает его значимым и актуальным.
\section{Лабораторная 4}
\subsection{Реализация:}

\section{Изучение основных концепций и инструментов:}
\section{Реализация системы искусственного интеллекта на Prolog:}
\subsection{Создание правил и логики вывода для принятия решений на основе базы знаний и онтологии.}

\section{Оценка и интерпретация результатов:}

\section{Заключение:}
\subsection{Описание преимуществ и потенциальных применений разработанной системы искусственного интеллекта на базе Prolog, баз знаний и онтологий.}
\nparagraph{Преимущества системы:}
\begin{itemize}
  \item Система позволяет пользователю получать рекомендации по выбору видеоигр на основе своих интересов и предпочтений, используя логический язык Prolog и базу знаний, содержащую информацию о различных играх серии Super Mario.
  \item Система использует декларативный подход к представлению знаний, что упрощает их описание и обновление. Система также способна проводить логический вывод и унификацию для поиска решения.
  \item Система демонстрирует возможности Prolog и семантических технологий для разработки систем искусственного интеллекта, таких как экспертные системы, рекомендательные системы, системы обработки естественного языка и другие
  \item Система имеет практическое значение для пользователей, которые хотят найти подходящую видеоигру из серии Super Mario
\end{itemize}
\nparagraph{Потенциальные применения системы:}
\begin{itemize}
  \item Система может быть использована для помощи пользователям в выборе видеоигр не только из серии Super Mario, но и из других жанров и франшиз, расширяя базу знаний или онтологию соответствующими данными.
  \item Система может быть интегрирована с другими приложениями и сервисами, такими как веб-сайты, мобильные приложения, голосовые ассистенты и социальные сети.
\end{itemize}
\end{document}
